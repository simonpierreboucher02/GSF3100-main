\documentclass{beamer}
\usepackage[utf8]{inputenc}
\usepackage{graphicx}
\usepackage{tikz}
\usetheme{default}

\usecolortheme{default}

\title[S08  Marché obligataire international]{Section 08 : Marché obligataire international}
\subtitle{GSF-3100 Marché des capitaux}
\author[SP. Boucher]{Simon-Pierre Boucher\inst{1}}
\institute[Université Laval]
{
  \inst{1}%
  Département de finance, assurance et immobilier\\
  Faculté des sciences de l'administration\\
  Université Laval}
\date[Automne 2021]{Automne 2021}

\begin{document}

\begin{frame}
\titlepage
\end{frame}

\begin{frame}{Introduction}
\begin{itemize}
\item Du point de vue d’un investisseur local, il est possible de classer les secteurs du marché obligataire international en:
\begin{itemize}
\item Marché interne (ou national), se subdivisant en: 
\begin{itemize}
\item Marché domestique (ou local);
\item Marché étranger.
\end{itemize}
\item Marché externe (aussi appelé marché international, marché euro-obligataire ou marché « offshore »).
\end{itemize}
\item L’objectif de cette présentation est d’introduire les principaux instruments financiers de ces secteurs. 
\end{itemize}
\end{frame}

\begin{frame}{Investir ou se financer internationalement}
\begin{block}{Avantages:}
\begin{itemize}
\item Plus grande diversification;
\item Plus d’opportunités d’investissement ou de financement;
\item Possibilités de tirer avantage d’un traitement fiscal ou d’une réglementation plus favorable.
\end{itemize}
\end{block}
\end{frame}

\begin{frame}{Investir ou se financer internationalement}
\begin{block}{Inconvénients:}
\begin{itemize}
\item Risques liés aux taux de change et aux pays;
\item Frais de transaction plus élevés et circulation moins immédiate des fonds;
\item Information plus volumineuse, plus dispersée et moins standardisée.
\end{itemize}
\end{block}
\end{frame}



\begin{frame}{Marché interne étranger}
\begin{block}{Obligation étrangère:}
\begin{itemize}
\item Titre négociable à moyen et long terme, émis sur un marché autre que celui de l’émetteur et normalement libellé en devises locales;
\item Placée par un syndicat bancaire, souvent dominé par les banques locales, principalement auprès des investisseurs locaux;
\item Les marchés du Canada, des États-Unis, du Japon, de la Grande-Bretagne, des Pays-Bas et de l’Espagne sont respectivement surnommés les marchés Maple, Yankee, Samurai, Bull-dog,Rembrandt et Matador.
\end{itemize}
\end{block}
\end{frame}

\begin{frame}{Marché externe}
\begin{block}{Euro-obligation:}
\begin{itemize}
\item Titre négociable à moyen et long terme émis en eurodevises simultanément dans plusieurs pays (excluant celui dont la monnaie sert à libeller l’emprunt);
\item Placée par un syndicat international de banques;
\item Bénéficie d’un statut fiscal privilégié car émis à l’extérieur des juridictions locales;
\item Eurodevises: Dépôts en monnaie convertible effectués dans des banques situées à l’extérieur du système monétaire (et législatif) de la devise considérée. Exemple: Eurodollar pour les É.-U.
\end{itemize}
\end{block}
\end{frame}

\begin{frame}{Marché externe}
\begin{block}{Certains instruments financiers euro-obligataires:}
\begin{itemize}
\item Obligation standard (Euro straight), sauf ayant un coupon annuel, souvent remboursable avant l’échéance et émise à 95 \% du temps dans l’une des dix devises les plus importantes;
\item Note à taux variable, souvent fonction du taux LIBOR (London interbank offered rate), pouvant inclure un plafond (cap) et/ou plancher (floor);
\item Obligation à double devises, avec les coupons et le principal payés en deux devises différentes;
\item Obligation à devise optionnelle (option currency bond), avec choix de devise pour l’émetteur ou l’investisseur; 
\end{itemize}
\end{block}
\begin{block}{Marché global (ou mondial): }
\begin{itemize}
\item Obligations offertes simultanément sur l’euromarché et un marché étranger (comme le marché \textbf{Yankee}).  
\end{itemize}
\end{block}
\end{frame}

\begin{frame}{Émetteurs sur le marché obligataire international}
\begin{itemize}
\item Gouvernements souverains: Gouvernements centraux des pays;
\item Gouvernements non-souverains: Entités gouvernementales non souveraines, comme les provinces ou états;
\item Agences supranationales: Entités formés de plusieurs gouvernements centraux à travers des traités internationaux, comme la Banque mondiale; 
\item Institutions financières;
\item Corporations. 
\end{itemize}
\end{frame}

\begin{frame}{Risque de crédit des pays}
\begin{block}{Critères influençant la cote de crédit des pays:}
\begin{itemize}
\item Risque politique;
\item Structure des revenus et de l’économie;
\item Potentiel de croissance économique;
\item Flexibilité fiscale;
\item Importance de la dette publique;
\item Stabilité des prix;
\item Flexibilité dans la balance des paiements;
\item Dette externe et liquidité.
\end{itemize}
\end{block}
\end{frame}

\begin{frame}{Institutions financières internationales}
\begin{block}{Banque mondiale (créée en 1944):}
\begin{itemize}
\item Banque d’investissement, propriété des gouvernements des pays membres, dont le mandat est de faciliter le développement économique et de réduire la pauvreté dans les pays moins avancés. 
\item Elle emprunte de façon importante sur les marchés des capitaux internationaux pour ensuite prêter à des pays en émergence à des taux favorables.  
\end{itemize}
\end{block}
\end{frame}
\begin{frame}{Institutions financières internationales}
\begin{block}{Fonds monétaire international (FMI, créé en 1944):}
\begin{itemize}
\item Institution de surveillance, financé par les quotes-parts de ses pays membres, ayant pour mission de superviser les politiques économiques de ses membres.
\item Il fournit également une aide financière temporaire conditionnelle aux pays connaissant des difficultés de balance de paiements et du financement à long terme dans le cadre de réformes s’inspirant des mécanismes du marché.  
\end{itemize}
\end{block}
\end{frame}
\begin{frame}{Institutions financières internationales}
\begin{block}{Banque des règlements internationaux (BRI, créée en 1930):}
\begin{itemize}
\item Banque qui place les devises et l’or des banques centrales, leur fournit une aide spéciale lors des crises financières et accueille des réunions de consultation à l’intention de leurs représentants sur des questions concernant le système financier international.
\item Elle est surnommée la banque centrale des banques centrales.
\item La Banque du Canada y a adhéré en 1970. 
\end{itemize}
\end{block}
\end{frame}
\end{document}