\documentclass{beamer}
\usepackage[utf8]{inputenc}
\usepackage{graphicx}
\usepackage{tikz}
\usetheme{default}

\usecolortheme{default}

\title[S05 Marché Monétaire]{Section 05 : Marché Monétaire}
\subtitle{GSF-3100 Marché des capitaux}
\author[SP. Boucher]{Simon-Pierre Boucher\inst{1}}
\institute[Université Laval]
{
  \inst{1}%
  Département de finance, assurance et immobilier\\
  Faculté des sciences de l'administration\\
  Université Laval}
\date[Automne 2021]{Automne 2021}

\begin{document}

\begin{frame}
\titlepage
\end{frame}

\begin{frame}{Introduction}
\begin{itemize}
\item Le marché monétaire est le marché où s’échange des actifs financiers avec une échéance courte (un an et moins).  
\item Le marché monétaire est composé des instruments suivants:
\begin{itemize}
\item Bon du Trésor
\item Papier commercial
\item Acceptation bancaire
\item Convention de rachat
\end{itemize}
\end{itemize}
\end{frame}

\begin{frame}{Bon du trésor}
\begin{itemize}
\item Le bon du trésor est un titre financier émis à courte échéance par les gouvernements et possédant les caractéristiques suivantes:
\begin{itemize}
\item Sans risque sur le marché domestique
\item Très liquide
\item Émis à escompte (sans coupon)
\item Dénomination en multiple de 1000 \$
\item Émis sur le marché primaire par adjudication (enchère).
\end{itemize}
\end{itemize}
\end{frame}

\begin{frame}{Bon du trésor canadien}
\begin{block}{Caractéristiques}
\begin{itemize}
\item Échéances : 3 mois,  6 mois et 12 mois
\item Cycle d’émission aux 2 semaines (mardi)
\item Procède par réouverture d’émissions
\item Convention de taux: Taux de rendement (nominal) sur une base de 365 jours par année.
\end{itemize}
\end{block}
\end{frame}

\begin{frame}{Bon du trésor canadien}
\begin{block}{Calcul taux de rendement}
\begin{align*}
Taux = \frac{D}{P} \times \frac{365}{t}=\frac{F-P}{P} \times \frac{365}{t}
\end{align*}
Sachant:
\begin{itemize}
\item $F$ représente la valeur nominal 
\item $P$ représente le prix du bon du trésor 
\item $t$ représente le temps avant l'échance 
\item $D$ représente l'escompte, soit la différence entre la valeur nominal et le prix $(F-P)$
\end{itemize}
\end{block}
\end{frame}

\begin{frame}{Bon du trésor canadien}
\begin{block}{Calcul taux de rendement}
Sachant les valeurs suivantes pour t :
\begin{table}[H]
\centering
\begin{tabular}{@{}ll@{}}
\toprule
\textbf{Échéance} & \multicolumn{1}{c}{\textbf{t}} \\ \midrule
\textbf{3 mois}   & 98 jours                       \\
\textbf{6 mois}   & 182 jours                      \\
\textbf{12 mois}  & 364 jours                      \\ \bottomrule
\end{tabular}
\end{table}
\end{block}
\end{frame}



\begin{frame}{Bon du Trésor Américain}
\begin{block}{Caractéristiques}
\\begin{itemize}
\item  Échéances : 1 mois, 3 mois,  6 mois et 12 mois
\item Cycle d’émission au semaine (lundi)
\item Procède par réouverture d’émissions
\item Convention de taux: Taux d’escompte (nominal) sur une base de 360 jours par année.
\end{itemize}
\end{block}
\end{frame}

\begin{frame}{Bon du Trésor Américain}
\begin{block}{Calcul taux de rendement}
\begin{align*}
Taux = \frac{D}{F} \times \frac{365}{t}=\frac{F-P}{F} \times \frac{360}{t}
\end{align*}
Sachant:
\begin{itemize}
\item $F$ représente la valeur nominal 
\item $P$ représente le prix du bon du trésor 
\item $t$ représente le temps avant l'échance 
\item $D$ représente l'escompte, soit la différence entre la valeur nominal et le prix $(F-P)$
\end{itemize}
\end{block}
\end{frame}

\begin{frame}{Bon du Trésor Américain}
\begin{block}{Calcul taux de rendement}
Sachant les valeurs suivantes pour t :
\begin{table}[H]
\centering
\begin{tabular}{@{}ll@{}}
\toprule
\textbf{Échéance} & \textbf{t} \\ \midrule
\textbf{1 mois}   & 28 jours   \\
\textbf{3 mois}   & 91 jours   \\
\textbf{6 mois}   & 182 jours  \\
\textbf{12 mois}  & 364 jours  \\ \bottomrule
\end{tabular}
\end{table}
\end{block}
\end{frame}

\begin{frame}{Papier commerciaux}
Billet émis par une corporation ayant une excellente cote de crédit.
\begin{block}{Caractéristiques}
\begin{itemize}
\item Normalement à escompte
\item Souvent endossé par une société-mère ou une marge de crédit bancaire
\item Échéance usuelle de 30 à 90 jours
\item Dénomination minimum de 50 000 \$
\item Peut comporter des clauses de rachat
\item Rendement supérieur à celui des bons du Trésor.
\end{itemize}
\end{block}
\end{frame}

\begin{frame}{Acceptation bancaire}
Billet garanti par une banque émis par une corporation n’ayant pas une cote de crédit suffisante pour émettre des papiers commerciaux.
\begin{block}{Caractéristiques}
\begin{itemize}
\item Frais de la banque de 0,25 \% à 0,75 \%
\item Normalement à escompte
\item Échéance usuelle de 30 à 90 jours
\item Dénomination minimum de 100 000 \$
\item Très liquide
\item Rendement supérieur à celui des bons du Trésor mais inférieur à celui du papier commercial.
\end{itemize}
\end{block}
\end{frame}


\begin{frame}{Convention de rachat}
Vente d’un titre financier avec la promesse de le racheter à une date future (fixée) au même prix plus un montant d’intérêt spécifié.
\begin{block}{Caractéristiques}
\begin{itemize}
\item La \textbf{garantie} est le titre financier acheté
\item Échéance usuelle: 1 à 364 jours
\item Très liquide
\item Taux d’intérêt fonction du loyer de l’argent
\item Souvent utilisée par les courtiers en valeurs mobilières pour financer leur inventaire.
\end{itemize}
\end{block}
\end{frame}


\begin{frame}{Convention de rachat}

\begin{block}{Instruments reliés}
\begin{itemize}
\item Vente à réméré (sell/buyback agreement): Lorsqu’il n’y a pas d’intérêt de verser mais le prix de rachat est différent du prix de vente.
\item Convention à un jour (overnight repo): Si l’échéance est d’une journée.
\item Convention ouverte (open repo): Sans échéance (se terminant à la demande d’une des parties).
\item Prêt de valeurs mobilières (securities lending agreement): Échange qui peut impliquer de la liquidité, mais aussi d’autres actifs financiers.
\end{itemize}
\end{block}
\end{frame}


\begin{frame}{Politique monétaire}

La politique monétaire vise à préserver la valeur de la monnaie en maintenant l’inflation à un niveau bas, stable et prévisible.  Le cadre de conduite de la politique monétaire canadienne s’appuie sur deux composantes principales qui fonctionnent de concert et se renforcent mutuellement : 
\begin{itemize}
\item la cible de maîtrise de l’inflation
\item le taux de change flottant
\end{itemize}
\end{frame}


\begin{frame}{Politique monétaire}

\begin{block}{Cible de maîtrise de l’inflation}
\begin{itemize}
\item La cible de maîtrise de l’inflation, qui est fixée à 2 \%, soit le point médian d’une fourchette qui va de 1 à 3 \%, constitue l’élément central du cadre de conduite de la politique monétaire du Canada
\end{itemize}
\end{block}

\begin{block}{Taux de change flottant du Canada}
\begin{itemize}
\item Le taux de change flottant du Canada permet à la Banque de mener une politique monétaire indépendante, qui est bien adaptée à la conjoncture économique au pays et axée sur l’atteinte de la cible d’inflation. 
\item Les variations du taux de change remplissent également un rôle d’amortisseur,  car elles aident l’économie à absorber les chocs externes et internes et à s’y ajuster.
\end{itemize}
\end{block}
\end{frame}

\end{document}